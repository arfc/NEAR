\documentclass{anstrans}
%%%%%%%%%%%%%%%%%%%%%%%%%%%%%%%%%%%
\title{Title in progress}
% Fuel Cycle of Non-Equilibrium Reactors with Enrichment and Core Loading Versatility
% I'd like to use NEAR if possible
\author{Nathan S. Ryan,$^{a, b}$ Jin Whan Bae,$^{a}$}

\institute{
$^{a}$Research and Test Reactor Physics Group, Oak Ridge National Laboratory,
Oak Ridge, TN
\and
$^{b}$Advanced Reactors and Fuel Cycles Group, University of Illinois Urbana-Champaign, Urbana, IL, nsryan2@illinois.edu
}

%%%% packages and definitions (optional)
\usepackage{graphicx} % allows inclusion of graphics
\usepackage{booktabs} % nice rules (thick lines) for tables
\usepackage{microtype} % improves typography for PDF
\usepackage{xspace}
\usepackage[acronym]{glossaries}
\usepackage{enumitem}

\setlist{topsep=1.5em, itemsep=0.002em}
\newcommand{\cyclus}{\textsc{Cyclus}\xspace} %
\newcommand{\cycamore}{\textsc{Cycamore}\xspace} %
\newcommand{\cymetric}{\textsc{Cymetric}\xspace} %
\newcommand{\ever}{\textsc{ever}\xspace} %
\newcommand{\clover}{\textsc{CLOVER}\xspace} %
\newcommand{\near}{\textsc{NEAR}\xspace} %
\newcommand{\SN}{S$_N$}
\renewcommand{\vec}[1]{\bm{#1}} %vector is bold italic
\newcommand{\vd}{\bm{\cdot}} % slightly bold vector dot
\newcommand{\grad}{\vec{\nabla}} % gradient
\newcommand{\ud}{\mathop{}\!\mathrm{d}} % upright derivative symbol

\makeglossaries


\begin{document}
\newacronym{ever}{bla}{\textbf{E}nrichment \textbf{V}ersatile non-\textbf{E}quilibrium \textbf{R}eactor}
\newacronym{clover}{CLOVER}{\textbf{C}ore \textbf{LO}ading \textbf{V}ersatile non-\textbf{E}quilibrium \textbf{R}eactor}
\newacronym{near}{NEAR}{\textbf{N}on-\textbf{E}quilibrium \textbf{A}rchetypal \textbf{R}eactor}
\section{Introduction}

In this work we will build alongside the work of Amanda Bachmann on OpenMCyclus \cite{bachmann_thesis_2023} to create greater flexibility in the \cyclus \cite{cyclus_intro} ecosystem for versatile simulation of non-equilibrium core reactors. We will create versatile archetypes that can be used to approximate specific designs instead of modeling a specific reactor. We will be comparing the gain in fidelity to the existing \cycamore reactor archetype in a series of baseline scenarios. Our two primary thrusts of this work are to enhance fidelity in depletion and fuel loading.

% say what fuel cycle factors you are studying and why to do this


When we translate these fuel cycle factors into real life scenarios, exogenous factors such as human behavior, regulation, and economic constraint can drive models to differing conclusions; therefore, we can increase the relevence of our models by incorporating modular and versatile archetypes.

\subsection{The \cyclus Ecosystem}
In this work, we will distinguish between the code \cyclus, and the collection of software \cyclus by labeling the latter the \cyclus ecosystem. The \cyclus ecosystem is comprised of a core set of codes: \cyclus, \cycamore, and \cymetric, developer maintained codes, and user-maintained codes.

The code \cyclus houses the base engine and several fundamental facility types, such as preditor and prey facilities. \cycamore adds in a series of modular nuclear fuel cycle facilities that users can deploy to create all manner of fuel cycles, while \cymetric contains code allowing users to analyze their \cyclus runs. Together these three tools comprise the modular core of the \cyclus ecosystem; outside this core are codes maintained by the \cyclus developers and third party codes, which give users additional functionality to create high fidelity fuel cycle simulations.

Many of the codes outside the core are third party archetypal facilities, and, as an agent-based code, \cyclus users can incorporate them in a plug-and-play fashion. When developers find an area of the fuel cycle not covered by the \cycamore archetypes, they can build their own to increase fidelity or expand their modeling capabilities.

In this work will create and demonstrate three \cyclus archetypes:
\begin{enumerate}
    \item \gls{ever}
    \item \gls{clover}
    \item \gls{near}
\end{enumerate}

NEAR contains the functionality of both EVER and CLOVER, but we will deploy
them separately to allow future users a simplified interface if they do not
want all of NEAR's functionality. As such, we will present the theory behind
EVER and CLOVER, but our results will include NEAR.


\section{Theory}


NEAR contains the functionality of both EVER and CLOVER, but we will deploy
them separately to allow future users a simplified interface if they do not
want all of NEAR's functionality. As such, we will present the theory behind
EVER and CLOVER, but our results will include NEAR.


\section{Preliminary Results}

Users of the \cycamore reactor archetype have the ability to change the inputs and outputs at various times in the simultaion. A user could use this capability to alter the fuel recipe to reflect the changing isotopics of a reacotr core, or the impact of different core configurations. We implemented these features in \gls{ever} and \gls{clover} to be explitict to compare with \cycamore


\subsection{Existing Enrichment Fidelity}

First
\gls{ever}


\subsection{Existing Core Loading Fidelity}

Yikes
\gls{clover}


\section{Conclusions}




\section{Acknowledgments}
% insert the neup?

% Copied from the appointment documents
This research was supported in part by an appointment to the Oak Ridge National Laboratory Research Student Internships Program, sponsored by the U.S. Department of Energy and administered by the Oak Ridge Institute for Science and Education.

% personalized
The authors would like to thank Eva Davidson and Madicken Munk for their role
in shaping the direction of this project, as well as (the wisconsin people?) and Kathryn Huff (?).

\bibliographystyle{ans}
\bibliography{bibliography}
\printglossary[type=\acronymtype]
\printglossary


% Lorem ipsum dolor sit amet, consectetur adipiscing elit. Curabitur faucibus
% erat sed nisi aliquet molestie. Etiam malesuada, sapien at lobortis lacinia,
% justo ante volutpat nunc, gravida commodo justo purus ut quam. Proin tincidunt
% sem quis dui condimentum rhoncus. Nulla ut libero est, ut sollicitudin ligula.
% Curabitur quam orci, aliquet dignissim feugiat eu, porta ac leo. Aenean in
% ipsum arcu. Duis tempus porttitor turpis, eu volutpat odio fringilla sit amet.
% Donec malesuada, arcu id porttitor mattis, arcu est molestie arcu, quis
% dignissim tellus justo nec sapien. Praesent pretium interdum odio ac varius.
% Suspendisse dui mauris, posuere in varius a, semper id dui. Suspendisse
% placerat, quam quis luctus aliquam, metus justo hendrerit massa, vel tempus
% ligula sem et purus.

% Lorem ipsum dolor sit amet, consectetur adipiscing elit. Curabitur faucibus
% erat sed nisi aliquet molestie.

% Equations look exceedingly pretty. Here is a 3-D, monoenergetic, steady-state
% transport equation with isotropic scattering and an isotropic extraneous source:
% \begin{subequations} \label{eqs:fullTransport}
% \begin{multline} \label{eq:fullTransportVol}
%   \vec{\Omega}\vd \grad \psi(\vec{x}, \vec{\Omega})
%   + \sigma(\vec{x}) \psi (\vec{x}, \vec{\Omega})
% \\ =
%   \frac{\sigma_s(\vec{x})}{4\pi} \int_{4\pi} \psi(\vec{x},\vec{\Omega}')
%   \ud\Omega' + \frac{q(\vec{x})}{4\pi}
%   \equiv \frac{1}{4\pi} Q(\vec{x}) \,,
% \end{multline}
% inside $\vec{x} \in V$, $\vec{\Omega} \in 4\pi$, with an incident boundary
% condition
% \begin{equation} \label{eq:fullTransportBndy}
%   \psi(\vec{x}, \vec{\Omega}) = \psi^b(\vec{x}, \vec{\Omega}) \,,
%  \quad \vec{x} \in \partial V, \ \vec{\Omega} \vd \vec{n} < 0\,.
% \end{equation}
% \end{subequations}

%%%%%%%%%%%%%%%%%%%%%%%%%%%%%%%%%%%%%%%%%%%%%%%%%%%%%%%%%%%%%%%%%%%%%%%%%%%%%%%%

% The user must manually capitalize initial letters of a subsection heading.

% For those who like equations in their papers, \LaTeX\ is a good choice. Here is
% an equation for the Marshak diffusion boundary condition:
% \begin{equation} \label{eq:marshak}
%   4 J^- = \phi + 2 D \vec{n} \vd \grad \phi \,.
% \end{equation}
% If we so choose, we can effortlessly reference the equation later.

% Another paragraph starts with Eq.~\eqref{eq:marshak} and sets $J^-$ to zero, a
% vacuum boundary condition:
% \begin{equation*}
%   0 = \phi + \frac{2}{3} \frac{1}{\sigma} \vec{n} \vd \grad \phi \,.
% \end{equation*}
% The extrapolation distance is $2/3$. A more detailed asymptotic analysis yields
% an extrapolation distance of about $0.71045$.

% Figure~\ref{fig:voltage} shows how a plot might conceivably look in your
% document. Always place figures after they are referenced so as not to throw
% off the reader. You can use symbols and different line styles to help
% differentiate your results, especially if they are printed in black and white.
% Note how Fig.~\ref{fig:voltage} uses dashed lines \verb|--| for the exact
% solution, solid lines \verb|-| for the new method's solutions, and dotted lines
% \verb|:| for existing inaccurate methods.
% \begin{figure}[ht] % replace 't' with 'b' to force it to be on the bottom
%   \centering
%   \includegraphics{example_figure}
%   \caption{Captions are flush with the left.}
%   \label{fig:voltage}
% \end{figure}

% Later on, we can include a table, even one that spans two columns such as
% Table~\ref{tab:widetable}.


%%%%%%%%%%%%%%%%%%%%%%%%%%%%%%%%%%%%%%%%
% \begin{table*}[htb]
%   \centering
%   \caption{Example of a Really Wide Table that Might Not Normally Fit in the Document}
%   \begin{tabular}{llllllllll}\toprule
%       & $\phi_T(0)$      & $\phi_T(10)$      & $\phi_T(20)$      &
%       $\phi_D(0)$      & $\phi_D(10)$      & $\phi_D(20)$      & $\rho$      &
%       $\varepsilon$      & $N_\text{it}$
% \\ \midrule
% $c=0.999$  & 0.9038 & 20.63 & 31.24 & 0.9087 & 20.63 & 31.23 & 0.2192 & $10^{-7}$ & 15
% \\
% $c=0.990$  & 0.3675 & 13.04 & 24.7 & 0.3696 & 13.04 & 24.69 & 0.2184 & $10^{-7}$ & 15
% \\
% $c=0.900$  & 0.009909 & 4.776 & 17.64 & 0.009984 & 4.786 & 17.63 & 0.2118 & $10^{-7}$ & 14
% \\
% $c=0.500$  & $6.069\times 10^{-5}$ & 2.212 & 15.53 & 6.213$\times 10^{-5}$ & 2.239 & 15.53 & 0.2068 & $10^{-7}$ & 13
% \\
% \bottomrule
% \end{tabular}
%   \label{tab:widetable}
% \end{table*}
%%%%%%%%%%%%%%%%%%%%%%%%%%%%%%%%%%%%%%%%
% Notice how the table reference uses a Roman numeral
% for its numbering scheme, whereas the figure reference uses an Arabic numeral.
% For one-column tables, use the \verb|table| environment; two-column tables use
% \verb|table*|. The same applies to figures.

%%%%%%%%%%%%%%%%%%%%%%%%%%%%%%%%%%%%%%%%%%%%%%%%%%%%%%%%%%%%%%%%%%%%%%%%%%%%%%%%

%%%%%%%%%%%%%%%%%%%%%%%%%%%%%%%%%%%%%%%%%%%%%%%%%%%%%%%%%%%%%%%%%%%%%%%%%%%%%%%%
% \appendix
% \section{Appendix}

% Numbering in the appendix is different:
% \begin{equation} \label{eq:appendix}
%   2 + 2 = 5\,.
% \end{equation}
% and another equation:
% \begin{equation} \label{eq:appendix2}
%   a + b = c\,.
% \end{equation}

%%%%%%%%%%%%%%%%%%%%%%%%%%%%%%%%%%%%%%%%%%%%%%%%%%%%%%%%%%%%%%%%%%%%%%%%%%%%%%%%

\end{document}

