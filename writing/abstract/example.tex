\documentclass{anstrans}
%%%%%%%%%%%%%%%%%%%%%%%%%%%%%%%%%%%
\title{Title in progress}
% Fuel Cycle of Non-Equilibrium Reactors with Enrichment and Core Loading Versatility
% I'd like to use NEAR if possible
\author{Nathan S. Ryan,$^{a, b}$ Jin Whan Bae,$^{a}$}

\institute{
$^{a}$Research and Test Reactor Physics Group, Oak Ridge National Laboratory,
Oak Ridge, TN
\and
$^{b}$Advanced Reactors and Fuel Cycles Group, University of Illinois Urbana-Champaign, Urbana, IL, nsryan2@illinois.edu
}

%%%% packages and definitions (optional)
\usepackage{graphicx} % allows inclusion of graphics
\usepackage{booktabs} % nice rules (thick lines) for tables
\usepackage{microtype} % improves typography for PDF
\usepackage{xspace}
\newcommand{\Cyclus}{\textsc{Cyclus}\xspace} %
\newcommand{\Cycamore}{\textsc{Cycamore}\xspace} %
\newcommand{\Cymetric}{\textsc{Cymetric}\xspace} %
\newcommand{\SN}{S$_N$}
\renewcommand{\vec}[1]{\bm{#1}} %vector is bold italic
\newcommand{\vd}{\bm{\cdot}} % slightly bold vector dot
\newcommand{\grad}{\vec{\nabla}} % gradient
\newcommand{\ud}{\mathop{}\!\mathrm{d}} % upright derivative symbol

\begin{document}
\section{Introduction}


\subsection{The \Cyclus Ecosystem}
In this work, we will distinguish between the code \Cyclus, and the collection of software \Cyclus by labeling the latter the \Cyclus ecosystem. The \Cyclus ecosystem is comprised of a core set of codes: \Cyclus, \Cycamore, and \Cymetric, developer maintained codes, and user-maintained codes.

The code \Cyclus houses the base engine and several fundamental facility types, such as preditor and prey facilities. \Cycamore adds in a series of modular nuclear fuel cycle facilities that users can deploy to create all manner of fuel cycles, while \Cymetric contains code allowing users to analyze their \Cyclus runs. Together these three tools comprise the modular core of the \Cyclus ecosystem; outside this core are codes maintained by the \Cyclus developers and third party codes, which give users additional functionality to create high fidelity fuel cycle simulations.

Many of the codes outside the core are third party archetypal facilities, and, as an agent-based code, \Cyclus users can incorporate them in a plug-and-play fashion. When developers find an area of the fuel cycle not covered by the \Cycamore archetypes, they can build their own to increase fidelity or expand their modeling capabilities.


\section{Theory}



\section{Preliminary Results}
Upon completion of this work, we will have created and deployed


\subsection{Existing Enrichment Fidelity}

First



\subsection{Existing Core Loading Fidelity}

Yikes



\section{Conclusions}
We need some final thoughts



\section{Acknowledgments}
Insert Acknowledgments

The authors would like to thank Eva Davidson and Madicken Munk for their role
in shaping the direction of this project, as well as (the wisconsin people?) and Kathryn Huff (?).

\bibliographystyle{ans}
\bibliography{bibliography}


% Lorem ipsum dolor sit amet, consectetur adipiscing elit. Curabitur faucibus
% erat sed nisi aliquet molestie. Etiam malesuada, sapien at lobortis lacinia,
% justo ante volutpat nunc, gravida commodo justo purus ut quam. Proin tincidunt
% sem quis dui condimentum rhoncus. Nulla ut libero est, ut sollicitudin ligula.
% Curabitur quam orci, aliquet dignissim feugiat eu, porta ac leo. Aenean in
% ipsum arcu. Duis tempus porttitor turpis, eu volutpat odio fringilla sit amet.
% Donec malesuada, arcu id porttitor mattis, arcu est molestie arcu, quis
% dignissim tellus justo nec sapien. Praesent pretium interdum odio ac varius.
% Suspendisse dui mauris, posuere in varius a, semper id dui. Suspendisse
% placerat, quam quis luctus aliquam, metus justo hendrerit massa, vel tempus
% ligula sem et purus.

% Lorem ipsum dolor sit amet, consectetur adipiscing elit. Curabitur faucibus
% erat sed nisi aliquet molestie.

% Equations look exceedingly pretty. Here is a 3-D, monoenergetic, steady-state
% transport equation with isotropic scattering and an isotropic extraneous source:
% \begin{subequations} \label{eqs:fullTransport}
% \begin{multline} \label{eq:fullTransportVol}
%   \vec{\Omega}\vd \grad \psi(\vec{x}, \vec{\Omega})
%   + \sigma(\vec{x}) \psi (\vec{x}, \vec{\Omega})
% \\ =
%   \frac{\sigma_s(\vec{x})}{4\pi} \int_{4\pi} \psi(\vec{x},\vec{\Omega}')
%   \ud\Omega' + \frac{q(\vec{x})}{4\pi}
%   \equiv \frac{1}{4\pi} Q(\vec{x}) \,,
% \end{multline}
% inside $\vec{x} \in V$, $\vec{\Omega} \in 4\pi$, with an incident boundary
% condition
% \begin{equation} \label{eq:fullTransportBndy}
%   \psi(\vec{x}, \vec{\Omega}) = \psi^b(\vec{x}, \vec{\Omega}) \,,
%  \quad \vec{x} \in \partial V, \ \vec{\Omega} \vd \vec{n} < 0\,.
% \end{equation}
% \end{subequations}

%%%%%%%%%%%%%%%%%%%%%%%%%%%%%%%%%%%%%%%%%%%%%%%%%%%%%%%%%%%%%%%%%%%%%%%%%%%%%%%%

% The user must manually capitalize initial letters of a subsection heading.

% For those who like equations in their papers, \LaTeX\ is a good choice. Here is
% an equation for the Marshak diffusion boundary condition:
% \begin{equation} \label{eq:marshak}
%   4 J^- = \phi + 2 D \vec{n} \vd \grad \phi \,.
% \end{equation}
% If we so choose, we can effortlessly reference the equation later.

% Another paragraph starts with Eq.~\eqref{eq:marshak} and sets $J^-$ to zero, a
% vacuum boundary condition:
% \begin{equation*}
%   0 = \phi + \frac{2}{3} \frac{1}{\sigma} \vec{n} \vd \grad \phi \,.
% \end{equation*}
% The extrapolation distance is $2/3$. A more detailed asymptotic analysis yields
% an extrapolation distance of about $0.71045$.

% Figure~\ref{fig:voltage} shows how a plot might conceivably look in your
% document. Always place figures after they are referenced so as not to throw
% off the reader. You can use symbols and different line styles to help
% differentiate your results, especially if they are printed in black and white.
% Note how Fig.~\ref{fig:voltage} uses dashed lines \verb|--| for the exact
% solution, solid lines \verb|-| for the new method's solutions, and dotted lines
% \verb|:| for existing inaccurate methods.
% \begin{figure}[ht] % replace 't' with 'b' to force it to be on the bottom
%   \centering
%   \includegraphics{example_figure}
%   \caption{Captions are flush with the left.}
%   \label{fig:voltage}
% \end{figure}

% Later on, we can include a table, even one that spans two columns such as
% Table~\ref{tab:widetable}.


%%%%%%%%%%%%%%%%%%%%%%%%%%%%%%%%%%%%%%%%
% \begin{table*}[htb]
%   \centering
%   \caption{Example of a Really Wide Table that Might Not Normally Fit in the Document}
%   \begin{tabular}{llllllllll}\toprule
%       & $\phi_T(0)$      & $\phi_T(10)$      & $\phi_T(20)$      &
%       $\phi_D(0)$      & $\phi_D(10)$      & $\phi_D(20)$      & $\rho$      &
%       $\varepsilon$      & $N_\text{it}$
% \\ \midrule
% $c=0.999$  & 0.9038 & 20.63 & 31.24 & 0.9087 & 20.63 & 31.23 & 0.2192 & $10^{-7}$ & 15
% \\
% $c=0.990$  & 0.3675 & 13.04 & 24.7 & 0.3696 & 13.04 & 24.69 & 0.2184 & $10^{-7}$ & 15
% \\
% $c=0.900$  & 0.009909 & 4.776 & 17.64 & 0.009984 & 4.786 & 17.63 & 0.2118 & $10^{-7}$ & 14
% \\
% $c=0.500$  & $6.069\times 10^{-5}$ & 2.212 & 15.53 & 6.213$\times 10^{-5}$ & 2.239 & 15.53 & 0.2068 & $10^{-7}$ & 13
% \\
% \bottomrule
% \end{tabular}
%   \label{tab:widetable}
% \end{table*}
%%%%%%%%%%%%%%%%%%%%%%%%%%%%%%%%%%%%%%%%
% Notice how the table reference uses a Roman numeral
% for its numbering scheme, whereas the figure reference uses an Arabic numeral.
% For one-column tables, use the \verb|table| environment; two-column tables use
% \verb|table*|. The same applies to figures.

%%%%%%%%%%%%%%%%%%%%%%%%%%%%%%%%%%%%%%%%%%%%%%%%%%%%%%%%%%%%%%%%%%%%%%%%%%%%%%%%

%%%%%%%%%%%%%%%%%%%%%%%%%%%%%%%%%%%%%%%%%%%%%%%%%%%%%%%%%%%%%%%%%%%%%%%%%%%%%%%%
% \appendix
% \section{Appendix}

% Numbering in the appendix is different:
% \begin{equation} \label{eq:appendix}
%   2 + 2 = 5\,.
% \end{equation}
% and another equation:
% \begin{equation} \label{eq:appendix2}
%   a + b = c\,.
% \end{equation}

%%%%%%%%%%%%%%%%%%%%%%%%%%%%%%%%%%%%%%%%%%%%%%%%%%%%%%%%%%%%%%%%%%%%%%%%%%%%%%%%

\end{document}

